\documentclass[12pt]{article}
\usepackage{ragged2e} % load the package for justification
\usepackage{hyperref}
\usepackage[utf8]{inputenc}
\usepackage{pgfplots}
\usepackage{tikz}
\usetikzlibrary{fadings}
\usepackage{filecontents}
\usepackage{multirow}
\usepackage{amsmath}
\pgfplotsset{width=10cm,compat=1.17}
\setlength{\parskip}{0.75em} % Set the space between paragraphs
\usepackage{setspace}
\setstretch{1.2} % Adjust the value as per your preference
\usepackage[margin=2cm]{geometry} % Adjust the margin
\setlength{\parindent}{0pt} % Adjust the value for starting paragraph
\usetikzlibrary{arrows.meta}
\usepackage{mdframed}
\usepackage{float}
\usepackage{geometry}

\usepackage{hyperref}

% to remove the hyperline rectangle
\hypersetup{
	colorlinks=true,
	linkcolor=black,
	urlcolor=blue
}

\usepackage{xcolor}
\usepackage{titlesec}
\usepackage{titletoc}
\usepackage{listings}
\usepackage{tcolorbox}
\usepackage{lipsum} % Example text package
\usepackage{fancyhdr} % Package for customizing headers and footers



% Define the orange color
\definecolor{myorange}{RGB}{255,65,0}
% Define a new color for "cherry" (dark red)
\definecolor{cherry}{RGB}{148,0,25}
\definecolor{codegreen}{rgb}{0,0.6,0}



%%%%%%%%%%%%%%%%%%%%%%%%%%%%%%%%%%%%%%%%%%%%%%%%%%%%%%%%%%%%%%%%%%%%%
% Apply the custom footer to all pages
\pagestyle{fancy}

% Redefine the header format
\fancyhead{}
\fancyhead[R]{\textcolor{orange!80!black}{\itshape\leftmark}}

\fancyhead[L]{\textcolor{black}{\thepage}}


% Redefine the footer format with a line before each footnote
\fancyfoot{}
\fancyfoot[C]{\footnotesize Marco Tan, McMaster University, Programming for Mechatronics - MECHTRON 2MP3. \footnoterule}

% Redefine the footnote rule
\renewcommand{\footnoterule}{\vspace*{-3pt}\noindent\rule{0.0\columnwidth}{0.4pt}\vspace*{2.6pt}}

% Set the header rule color to orange
\renewcommand{\headrule}{\color{orange!80!black}\hrule width\headwidth height\headrulewidth \vskip-\headrulewidth}

% Set the footer rule color to orange (optional)
\renewcommand{\footrule}{\color{black}\hrule width\headwidth height\headrulewidth \vskip-\headrulewidth}

%%%%%%%%%%%%%%%%%%%%%%%%%%%%%%%%%%%%%%%%%%%%%%%%%%%%%%%%%%%%%%%%%%%%%


% Set the color for the section headings
\titleformat{\section}
{\normalfont\Large\bfseries\color{orange!80!black}}{\thesection}{1em}{}

% Set the color for the subsection headings
\titleformat{\subsection}
{\normalfont\large\bfseries\color{orange!80!black}}{\thesubsection}{1em}{}

% Set the color for the subsubsection headings
\titleformat{\subsubsection}
{\normalfont\normalsize\bfseries\color{orange!80!black}}{\thesubsubsection}{1em}{}


%%%%%%%%%%%%%%%%%%%%%%%%%%%%%%%%%%%%%%%%%%%%%%%%%%%%%%%%%%%%%%%%%%%%%
% Set the color for the table of contents
\titlecontents{section}
[1.5em]{\color{orange!80!black}}
{\contentslabel{1.5em}}
{}{\titlerule*[0.5pc]{.}\contentspage}

% Set the color for the subsections in the table of contents
\titlecontents{subsection}
[3.8em]{\color{orange!80!black}}
{\contentslabel{2.3em}}
{}{\titlerule*[0.5pc]{.}\contentspage}

% Set the color for the subsubsections in the table of contents
\titlecontents{subsubsection}
[6em]{\color{orange!80!black}}
{\contentslabel{3em}}
{}{\titlerule*[0.5pc]{.}\contentspage}


%%%%%%%%%%%%%%%%%%%%%%%%%%%%%%%%%%%%%%%%%%%%%%%%%%%%%%%%%%%%%%%%%%%%%
% set a format for the codes inside a box with C format
\lstset{
	language=C,
	basicstyle=\ttfamily,
	backgroundcolor=\color{blue!5},
	keywordstyle=\color{blue},
	commentstyle=\color{codegreen},
	stringstyle=\color{red},
	showstringspaces=false,
	breaklines=true,
	frame=single,
	rulecolor=\color{lightgray!35}, % Set the color of the frame
	numbers=none,
	numberstyle=\tiny,
	numbersep=5pt,
	tabsize=1,
	morekeywords={include},
	alsoletter={\#},
	otherkeywords={\#}
}




%\input listings.tex



% Define a command for inline code snippets with a colored and rounded box
\newtcbox{\codebox}[1][gray]{on line, boxrule=0.2pt, colback=blue!5, colframe=#1, fontupper=\color{cherry}\ttfamily, arc=2pt, boxsep=0pt, left=2pt, right=2pt, top=3pt, bottom=2pt}




\tikzset{%
	every neuron/.style={
		circle,
		draw,
		minimum size=1cm
	},
	neuron missing/.style={
		draw=none, 
		scale=4,
		text height=0.333cm,
		execute at begin node=\color{black}$\vdots$
	},
}



%%%%%%%%%%%%%%%%%%%%%%%%%%%%%%%%%%%%%%%%%%%%%%%%%%%%%%%%%%%%%%%%%%%%%

% Define a new tcolorbox style with default options
\tcbset{
	myboxstyle/.style={
		colback=orange!10,
		colframe=orange!80!black,
	}
}

% Define a new tcolorbox style with default options to print the output with terminal style


\tcbset{
	myboxstyleTerminal/.style={
		colback=blue!5,
		frame empty, % Set frame to empty to remove the fram
	}
}

\mdfdefinestyle{myboxstyleTerminal1}{
	backgroundcolor=blue!5,
	hidealllines=true, % Remove all lines (frame)
	leftline=false,     % Add a left line
}


\begin{document}
	
	\justifying
	
	\begin{center}
		\textbf{{\large MECHTRON 2MP3 - Porgramming for Mechatronics}}
		
		\textbf{Developing a Basic Genetic Optimization Algorithm in C} 
		
		Marco Tan, 400433483
	\end{center}
	

	
	
	
	\section{Introduction}
	
	
	\section{Programming Genetic Algorithm}

	\subsection{Implementation (12 points)}
	
	Files you have downloaded are:
	
	\begin{itemize}
		\item \codebox{OF.c} (Do not change anything in it)
		\item \codebox{functions.h} (Do not change anything in it)
		\item \codebox{GA.c}
		\item \codebox{functions.c}
	\end{itemize}

		
		\begin{lstlisting}
			#include <math.h>
			#include "functions.h"
			
			double Objective_function(int NUM_VARIABLES, double x[NUM_VARIABLES])
			{
				double sum1 = 0.0, sum2 = 0.0;
				for (int i = 0; i < NUM_VARIABLES; i++)
				{
					sum1 += x[i] * x[i];
					sum2 += cos(2.0 * M_PI * x[i]);
				}
				return -20.0 * exp(-0.2 * sqrt(sum1 / NUM_VARIABLES)) - exp(sum2 / NUM_VARIABLES) + 20.0 + M_E;
			}
		\end{lstlisting}
	
	

	
	
	The number of decision variables in the optimization problem is set to 2, with upper and lower bounds for both decision variables set to +5 and -5, respectively:
	
	\begin{lstlisting}
		int NUM_VARIABLES = 2;
		double Lbound[] = {-5.0, -5.0};
		double Ubound[] = {5.0, 5.0};
	\end{lstlisting}
	
	
	\begin{mdframed}[style=myboxstyleTerminal1]
		\begin{verbatim}
			Genetic Algorithm is initiated.
			------------------------------------------------
			The number of variables: 2
			Lower bounds: [-5.000000, -5.000000]
			Upper bounds: [5.000000, 5.000000]
			
			Population Size:   100
			Max Generations:   10000
			Crossover Rate:    0.500000
			Mutation Rate:     0.100000
			Stopping criteria: 0.0000000000000001
			
			Results
			------------------------------------------------
			CPU time: 1.143935 seconds
			Best solution found: (-0.0001323239496775, -0.0000225231135396)
			Best fitness: 0.0003801313729501
		\end{verbatim}
	\end{mdframed}

    \begin{table}[h!]
        \small
        \caption{Results with Crossover Rate = 0.5 and Mutation Rate = 0.05, Stop Parameter = 0}
        \label{table:1}
        \centering
        \begin{tabular}{c c c c c c}
            \hline
            Pop Size & Max Gen & \multicolumn{3}{c}{Best Solution} & CPU time (Sec) \\
            & & $x_1$ & $x_2$ & Fitness & \\
            \hline
            10    & 100 & 0.1128262677755334  & 0.0167843350287455  & 1.574235    & 0.0001200 \\
            100   & 100 & 0.0157086155450479  & -0.0018225633547750 & 10.110173   & 0.0012430 \\
            1000  & 100 & 0.0021972204568783  & 0.0010652071801323  & 141.517785  & 0.0156620 \\
            10000 & 100 & -0.0000612600706802 & -0.0000696838833711 & 3792.785006 & 0.1752030 \\
            \hline
            1000 & 1000    & -0.0001671514474637 & -0.0001416751184227 & 1030.382844   & 0.1493740 \\
            1000 & 10000   & -0.0000269873999184 & -0.0000024517066786 & 12875.682388  & 1.5293810 \\
            1000 & 100000  & -0.0000051525421458 & 0.0000009662471712  & 53390.879056  & 14.9062560 \\
            1000 & 1000000 & 0.0000001234002411  & 0.0000001466833055  & 648436.305077 & 149.0835510 \\
            \hline
		\end{tabular}
	\end{table}

	\begin{table}[h!]
        \small
		\caption{Results with Crossover Rate = 0.5 and Mutation Rate = 0.05, Stop Parameter = 1e-16}
		\label{table:1}
		\centering
		\begin{tabular}{c c c c c c c}
			\hline
			Pop Size & Max Gen & \multicolumn{3}{c}{Best Solution} & CPU time (Sec) & Gens \\
			& & $x_1$ & $x_2$ & Fitness & \\
			\hline
			10    & 100 & 0.0337652373284865  & 0.2125752369000464  & 0.672888    & 0.0000770 & 64 \\
			100   & 100 & 0.0300162076158532  & 0.0395693560315156  & 4.876065    & 0.0001170 & 9 \\
			1000  & 100 & -0.0178259168834547 & 0.0017210398808682  & 16.899683   & 0.0026420 & 20 \\
			10000 & 100 & -0.0018487754286491 & 0.0018150242985291  & 133.197654  & 0.0270060 & 17 \\
			\hline
			1000 & 1000    & -0.0003592879513086 & -0.0000489712693028 & 970.763295    & 0.0200700 & 155 \\
			1000 & 10000   & 0.0000723614357749  & 0.0000571995042531  & 3815.094110   & 0.4947670 & 3761 \\
			1000 & 100000  & 0.0000153691507947  & -0.0000033271498996 & 21985.724392  & 2.2993190 & 17615 \\
			1000 & 1000000 & 0.0000002072192729  & 0.0000007334165275  & 316892.930285 & 13.9982390 & 106598 \\
			\hline
		\end{tabular}
	\end{table}

    \newpage

    With the way I implemented the genetic algorithm, we only stop running the GA if we've reached the stop condition for a consecutive number of generations. The way I calculated how many generations we have to consecutively reach the stop condition is \verb|MAX_GENERATIONS * 0.05|. As we can see, we get significantly better fitness if we run without a stop parameter versus when we do run with a stop parameter. This is because when we run the code without a stop parameter, we guarantee that the algorithm will run for the maximum number of generations allowed, which will converge our values closer to the global minimum. However, with a stop parameter, it is not guaranteed that our algorithm will run for that long, and so we get less fit values overall.

    I also implemented elitism in my GA to see if it would impact performance in terms of execution speed and our overall fitness. Elitism is a strategy in GAs in which we allocate a set amount of space for the most fit individuals in a population. We allow the fittest individuals a guaranteed spot in the next generation, which should theoretically increase the population's overall fitness. In theory, this should help us converge towards the optimized value. Here's a table with elitism enabled in the code:

    \begin{table}[h!]
        \small
		\caption{
    Results with Crossover Rate = 0.5 and Mutation Rate = 0.05, Stop Parameter = 0, With Elitism, Elitism Rate = 0.1}
		\label{table:1}
		\centering
		\begin{tabular}{c c c c c c}
			\hline
			Pop Size & Max Gen & \multicolumn{3}{c}{Best Solution} & CPU time (Sec) \\
			& & $x_1$ & $x_2$ & Fitness & \\
			\hline
			10    & 100 & -0.5699467731499794 & -0.0808229227926685 & 0.299646    & 0.0007130 \\
			100   & 100 & 0.0166529440398575  & -0.0265371683177245 & 8.727955    & 0.0487070 \\
			1000  & 100 & 0.0073487148654401  & -0.0035749678516641 & 40.172230   & 4.4630050 \\ 
			10000 & 100 & -0.0002649822273595 & 0.0002302718349876  & 1002.788198 & 369.5191410 \\ 
			\hline
			1000 & 1000    & 0.0002504023724468  & -0.0000487011857562 & 1380.741144  & 38.1626000 \\
			1000 & 10000   & -0.0000081094913220 & 0.0000106659717902  & 25705.486871 & 388.3493910 \\
			\hline
		\end{tabular}
	\end{table}

    \begin{table}[h!]
        \small
		\caption{
    Results with Crossover Rate = 0.5 and Mutation Rate = 0.05, Stop Parameter = 1e-16, With Elitism, Elitism Rate = 0.1}
		\label{table:1}
		\centering
		\begin{tabular}{c c c c c c c}
			\hline
			Pop Size & Max Gen & \multicolumn{3}{c}{Best Solution} & CPU time (Sec) & Gens \\
			& & $x_1$ & $x_2$ & Fitness & \\
			\hline
			10    & 100 & 0.2661389276693296  & 0.2415573062568699  & 0.365861    & 0.0003410 & 48 \\
			100   & 100 & -0.0376040325675184 & -0.1676733396796850 & 0.929195    & 0.0047660 & 12  \\
			1000  & 100 & 0.0048563512996100  & 0.0050673889951156  & 47.249322   & 0.7756500 & 21 \\
			10000 & 100 & 0.0009297556248171  & -0.0024037784907982 & 133.911033  & 75.8202890 & 21 \\
			\hline
			1000 & 1000    & 0.0030676205656812  & 0.0051449728222313  & 55.869929    & 2.0167950 & 54 \\
			1000 & 10000   & -0.0000632251613135 & -0.0000023585744211 & 5553.743165  & 58.5307230 & 1534 \\
			\hline
		\end{tabular}
	\end{table}

    \newpage

    (This would be a lot faster if I made this with parallelism in OpenMP in mind, but for now, this is fine.)

    Here, I compiled my code with the \verb|-D ELITISM| compile flag to tell the compile to define \verb|ELITISM|. I have a bunch of preprocessor directives in my code that check this definition and will include sections for implementing elitism when it is defined.

    As we can see, elitism doesn't necessarily make the results better since it's still dependent on the randomness of our initial population. This means that there's still a chance that we start our population off with a local minimum and because elitism guarantees the most fit individuals continue onward, we propagate that local minimum further. We only get out of that local minimum once mutation introduces enough genetic diversity to get us out. We also see that the time complexity has gone significantly up. I'm sure I could've optimized this further to make it faster (like using a parallel merge sort instead of C's \verb|qsort|), but this is fine for now.

    There were other strategies that I wanted to explore as well, such as repopulation (where we kill lower-fitness individuals in the population and replace them with new, randomly generated individuals) or the idea of using a circular genome for individuals in our population, but for now, this is sufficient. I'm hoping that maybe later, I could write my genetic algorithm (without template files) to further explore these concepts.
    
    \newpage
	
	\subsection{Report and \codebox{Makefile} (3 points)}
	
	The following is the contents of my \verb|Makefile|.

    \begin{lstlisting}
CC=gcc
CFLAGS=-lm -Ofast
DFLAGS=-Wall -Wextra -W -pedantic -O0 -g -pg -lm
ELITISM=

DEPS = functions.h
OBJS = GA.o OF.o functions.o

%.o: %.c $(DEPS)
	   $(CC) -c -o $@ $< $(CFLAGS)

GA: $(OBJS)
	   $(CC) -o $@ $^ $(CFLAGS)

.PHONY: clean

clean:
	   rm -f *.o GA
    \end{lstlisting}

    \begin{itemize}
        \item \verb|CC| defines the compiler we're using.
        \item \verb|CFLAGS| is all our compile flags.
        \begin{itemize}
            \item \verb|-lm| includes \verb|<math.h>|
            \item \verb|-Ofast| tells GCC to optimize the code as much as possible, for a little performance boost.
        \end{itemize}
        \item \verb|DFLAGS| are the compile flags I use when I'm still debugging the code.
        \begin{itemize}
            \item \verb|-Wall, -Wextra, -W| are all warning flags to enable warnings that may be important to me during compile time.
            \item \verb|-pedantic| ensures that I'm adhering to ANSI C. It doesn't matter much, but it's nice to have.
            \item \verb|-O0| ensures that the compiler does not optimize anything, such that the output when we debug is exactly what we expect.
            \item \verb|-g| is the flag we use if we want the binary to be compatible with \verb|gdb|.
            \item \verb|-pg| is the flag we use if we want the binary to be compatible with \verb|gprog|.
            \item \verb|-lm| is same as above, including \verb|<math.h>|.
        \end{itemize}
        \item \verb|ELITISM| can be filled with a \verb|-D ELITISM| flag if you want to compile the program with elitism.
    \end{itemize}

    Our \verb|DEPS| variable lists all header file dependencies that this program requires. \verb|OBJS| is a variable listing all the object files we need to compile before we can link them to the main program. \\

    The first declaration will compile all source code files in that directory, into object files (\verb|*.o|). The second declaration will gather the object files together and link them into the final output executable, \verb|GA|. \\

    \verb|.PHONY| states that the following declaration isn't to compile files, but as a helpful script we could run. I defined a \verb|clean| to clean the directory of the binary and object files if I wish to rebuild the code completely.
 
	\subsection{Improving the Performance - Bonus (+1 points)}

    Using \verb|gprof|, I realized that the most time-consuming task was the crossover function. My old implementation used a \verb|for| loop, iterating through a list of fitness and summing them successively so I could pick parents based on a roulette wheel selection algorithm. However, that would be $\mathcal{O}(n)$ time complexity and since I have to do it twice (because I have to pick two parents), it makes the code significantly slower. \\

    Instead of this, I sum up the probabilities beforehand and use a binary search to pick parents during the crossover. Binary search essentially divides a sorted list into two over and over again, narrowing the search space that contains the value we want until we get to the value. Binary search has a time complexity of $\mathcal{O}(\log n)$, which is significantly better performance than the iterative \verb|for| loop.
 
	\begin{table}[h!]
        \small
		\caption{Results with Crossover Rate = 0.5 and Mutation Rate = 0.2, Stop Parameter = 0, No Elitism}
		\label{table:1}
		\centering
		\begin{tabular}{c c c c c c}
			\hline
			Pop Size & Max Gen & \multicolumn{3}{c}{Best Solution} & CPU time (Sec) \\
			& & $x_1$ & $x_2$ & Fitness & \\
			\hline
			10    & 100  & 0.0232250546213351  & 0.0016808952212708  & 12.252658    & 0.0001190 \\
			100   & 100  & 0.0062705506599841  & -0.0028653233325411 & 48.155814    & 0.0012490 \\
			1000  & 100  & 0.0004579941744254  & -0.0000161421485325 & 647.887689   & 0.0150990 \\
			10000 & 100  & 0.0001305690967159  & 0.0000147777609598  & 2680.083076  & 0.1791700 \\
			\hline
			1000 & 1000    & 0.0000298605300628  & -0.0000156904570829 & 10369.297786  & 0.1543050 \\
			1000 & 10000   & 0.0000148243270885  & -0.0000003376044333 & 23284.917118  & 1.4530580 \\
			1000 & 100000  & -0.0000007892958829 & -0.0000000116415322 & 308228.189286 & 14.5638270 \\
			1000 & 1000000 & -0.0000000116415322 & -0.0000000349245965 & 707373.784597 & 145.2967410 \\
			\hline
		\end{tabular}
	\end{table}

    \begin{table}[h!]
        \small
		\caption{Results with Crossover Rate = 0.5 and Mutation Rate = 0.2, Stop Parameter = 1e-16, No Elitism}
		\label{table:1}
		\centering
		\begin{tabular}{c c c c c c c}
			\hline
			Pop Size & Max Gen & \multicolumn{3}{c}{Best Solution} & CPU time (Sec) & Gens \\
			& & $x_1$ & $x_2$ & Fitness & \\
			\hline
			10    & 100  & -0.8474522437189949 & 0.1049383194674451  & 0.331737     & 0.0000220 & 7 \\
			100   & 100  & -0.0149111985298394 & -0.0155445956697431 & 13.653199    & 0.0005510 & 42 \\
			1000  & 100  & -0.0012676022021409 & -0.0009034783583619 & 223.797324   & 0.0022020 & 15 \\
			10000 & 100  & 0.0004406436348523  & 0.0003503332847501  & 624.348251   & 0.0243050 & 14 \\
			\hline
			1000 & 1000    & 0.0000673555769337  & -0.0001193000935569 & 2570.712165   & 0.0768130 & 530 \\
			1000 & 10000   & -0.0000805058516935 & 0.0000638631172780  & 3425.462826   & 0.1676250 & 1144 \\
			1000 & 100000  & 0.0000046216882783  & -0.0000070151872968 & 40383.104333  & 2.1756900 & 14654 \\
			1000 & 1000000 & -0.0000001885928214 & -0.0000007566995928 & 311938.809237 & 33.3154300 & 228397 \\
			\hline
		\end{tabular}
	\end{table}
	
\end{document}